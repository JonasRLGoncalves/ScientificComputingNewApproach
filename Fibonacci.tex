\documentclass{article}
\usepackage[utf8]{inputenc}

\title{Fibonacci Recurrence}
\author{Jonas Gonçalves}
\date{October 2019}

\begin{document}

\maketitle

\section{Introduction}
    In mathematics, the Fibonacci numbers, commonly denoted \(F_n\) form a sequence, called the Fibonacci sequence, such that each number is the sum of the two preceding ones, starting from 0 and 1. That is,

        \[F_0 = 0 , F_1 = 1\]
    
        \begin{center}
            and
        \end{center}
    
        \begin{equation}
            F_n = F_{n-1} + F_{n-2}, \forall n > 1
        \end{equation}

\section{Exercise}
    Resolve the recurrence of fibonacci sequence.
\section{Linear}
    First, we'll write the characteristic polynomial of the recurrence
    \[F_n = F_{n-1} + F_{n-2} \Longrightarrow x^2 - x - 1 = 0\]
    The roots of this equation are:
    \[a_1 = \frac{1+\sqrt{5}}{2}, a_2 = \frac{1-\sqrt{5}}{2}\]
    We have that any expression on form show satisfies the recurssion:
    \[F_n = c_1\bigg(\frac{1+\sqrt{5}}{2}\bigg)^n + c_2\bigg(\frac{1-\sqrt{5}}{2}\bigg)^n\]
    Now, we only need to find \(c_1\) and \(c_2\) to satisfie initial conditions:
    \[n = 2 \Longrightarrow F_2 = c_1\bigg(\frac{1+\sqrt{5}}{2}\bigg)^2 + c_2\bigg(\frac{1-\sqrt{5}}{2}\bigg)^2 = 1\]
    Thus,
    \[c_1\bigg(\frac{1+\sqrt{5}}{2}\bigg)^2 + c_2\bigg(\frac{1-\sqrt{5}}{2}\bigg)^2 = 1\]
    And,
    \[n = 3 \Longrightarrow F_3 = c_1\bigg(\frac{1+\sqrt{5}}{2}\bigg)^3 + c_2\bigg(\frac{1-\sqrt{5}}{2}\bigg)^3 = 2\]
    \[c_1\bigg(\frac{1+\sqrt{5}}{2}\bigg)^3 + c_2\bigg(\frac{1-\sqrt{5}}{2}\bigg)^3 = 2\]
    Solving this system:
    \[c_1 = \frac{1}{\sqrt{5}}, c2 = \frac{-1}{\sqrt{5}}\]
    \begin{equation}
        F_n = \frac{1}{\sqrt{5}}\bigg(\frac{1+\sqrt{5}}{2}\bigg)^n - \frac{1}{\sqrt{5}}\bigg(\frac{1-\sqrt{5}}{2}\bigg)^n
    \end{equation}
\section{Generating Functions}
    \[F_n = F_{n-1} + F_{n-2}\]
    First, we  begin by defining the generating function for the Fibonacci numbers as the formal power series whose coefficients are the Fibonacci numbers themselves.
    \[F\textit{(x)} = \sum_{k=0}^\infty a_kx^k = \sum_{k=0}^\infty F_kx^k\]
    \[F\textit{(x)} = F\textit{(0)} + \sum_{k=1}^\infty F_kx^k = F\textit{(1)} + \sum_{k=2}^\infty F_kx^k = 1  + \sum_{k=2}^\infty F_kx^k \]
    \[F\textit{(x)} = x + \sum_{k=2}^\infty (F_{k-1} + F_{k-2})x^k = 1 + \sum_{k=2}^\infty F_{k-1}x^k + \sum_{k=2}^\infty F_{k-2}x^k\]
    \[F\textit{(x)} = x + x\sum_{k=0}^\infty F_kx^k + x^2\sum_{k=0}^\infty F_kx^k\]
    \[F\textit{(x)}(1 - x - x^2) = x \Longrightarrow F\textit{(x)} = \frac{x}{1 - x - x^2}\]
    We know that the roots of \(x^2 + x - 1\) are:
    \[r_1 = -\frac{1+\sqrt{5}}{2},r_2 = -\frac{1-\sqrt{5}}{2} \]
    \[ 1 - x - x^2 = (x-r_1)(x-r_2) = \bigg(x-\frac{1+\sqrt{5}}{2}\bigg)\bigg(x-\frac{1-\sqrt{5}}{2}\bigg)\]
    Thus,
    \[F\textit{(x)} = \frac{x}{1 - x - x^2} = \frac{1}{\sqrt{5}}\bigg(\frac{1}{x-r_1} -\frac{1}{x-r_2}\bigg)\]
    Let,
    \[\alpha = -r1 = \frac{1+\sqrt{5}}{2}, \beta = -r_2 = \frac{1-\sqrt{5}}{2} \]
    \[F\textit{(x)} = \frac{1}{\sqrt{5}}\bigg(\frac{1}{x+\alpha} -\frac{1}{x+\beta}\bigg) = \frac{1}{\sqrt{5}}\bigg(\sum_{k=0}^\infty (\alpha^k-\beta^k)x^k\bigg)\]
    Thus,
    \begin{equation}
        F\textit{(x)} = \frac{1}{\sqrt{5}}(\alpha^n-\beta^n) = \frac{1}{\sqrt{5}}\bigg\{\bigg(\frac{1+\sqrt{5}}{2}\bigg)^n-\bigg(\frac{1-\sqrt{5}}{2}\bigg)^n\bigg\}
    \end{equation}
    
\section{References}
    \begin{itemize}
        \item[\textbf{[1]}] https://en.wikipedia.org/wiki/Fibonacci\_number
        \item[\textbf{[2]}] https://www.wikihow.com/Solve-Recurrence-Relations
        \item[\textbf{[3]}] https://austinrochford.com/posts/2013-11-01-generating-functions-and-fibonacci-numbers.html
        \end{itemize}
\end{document}
