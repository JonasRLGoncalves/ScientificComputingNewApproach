\documentclass{article}
\usepackage[utf8]{inputenc}

\title{Artificial Intelligence, Machine Learning and Deep Learning}
\author{Jonas Gonçalves}
\date{January 2020}

\begin{document}

\maketitle

\section{Introduction}
    
    \textbf{Artificial Intelligence}, \textbf{Machine Learning} and \textbf{Deep Learning} are three hot buzzwords right now. The perception that they are the same thing could lead to some confusion. Those terms are popular when talking about topics as Big Data, analytics, and new technologies that are changing the world.
    
    The term Artificial Intelligence is more familiar, people are used to hearing this on several popular movies and series. But, recently, people started others terms like "Machine Learning" and "Deep Learning", many times being used interchangeably with Artificial Intelligence. As a result, the difference between those three words can be very unclear.
    
\section{Artificial Intelligence}

    \textbf{Artificial Intelligence} (\textbf{AI}) is the concept of intelligence demonstrated by machines, in contrast to human intelligence. The field can be defined as the study of ``\textbf{intelligent agents}", devices that perceives its environment and takes actions that maximize its chance of successfully achieving its goals.
    
    AI includes actions like planning, understanding language, recognizing objects and sounds, learning, and problem-solving. AI is about decision making, the ability of acquiring and applying knowledge.
    
    We can sort it in two categories, general and narrow. The general categories would resemble human intelligence, including capacities like those mentioned above. Narrow only exhibits some aspects, and can do those fairly well, but lacks in other areas. An example of narrow AI would be an agent that’s great at recognizing images, but nothing else.
    
\section{Machine Learning}

    \textbf{Machine Learning} (\textbf{ML}) is the scientific study of algorithms and statistical models that computer systems can use to perform a specific task without using explicit instructions, relying on patterns and inference instead.  Achieving an AI without the use of ML is possible, but this would require immense amounts of code lines, with specific rules and complex decisions-trees.
    
    Machine learning aims to increase its accuracy, it's the simple concept of a machine taking data and learning from it, not caring about it's success rate. Instead of coding specific instructions to accomplish a particular task, ML is a way of “training” an algorithm so that it can learn how. This “training” involves utilizing large amounts of data to the algorithm and allowing the algorithm to adjust itself and improve.
    

\section{Deep Learning}

    \textbf{Deep Learning} (\textbf{DL}) is part of a family of Machine Learning methods based on \textbf{Artificial Neural Networks} (\textbf{ANN}) with representation learning. So, it's one of many approaches to machine learning.
    
    Those neural networks, ANN, are computing systems vaguely inspired by the biological neural networks that constitute animal brains.  Such systems "learn" to perform tasks by considering examples, generally without being programmed with task-specific rules.
    
    There are “neurons” like structures, in ANNs, which have discrete layers and connections to other “neurons”. Each layer selects a specific feature to learn, such as "cat" or "no cat", in image recognition. This layering that gives Deep Learning its name, depth is created by using multiple layers as opposed to a single layer.

\section{Bibliography}

    \begin{itemize}
        \item[1.] https://medium.com/iotforall/the-difference-between-artificial-intelligence-machine-learning-and-deep-learning-3aa67bff5991\#744a
        \item[2.] https://www.geeksforgeeks.org/difference-between-machine-learning-and-artificial-intelligence/
        \item[3.] https://www.forbes.com/sites/bernardmarr/2016/12/06/what-is-the-difference-between-artificial-intelligence-and-machine-learning/\#63a832b2742b
        \item[4.] https://en.wikipedia.org/wiki/Artificial\_intelligence
        \item[5.] https://en.wikipedia.org/wiki/Machine\_learning
        \item[6.] https://en.wikipedia.org/wiki/Artificial\_neural\_network
        
    \end{itemize}

\end{document}
