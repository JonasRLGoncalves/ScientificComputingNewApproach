\documentclass{article}
\usepackage[utf8]{inputenc}

\title{Operadores}
\author{Jonas Gonçalves}
\date{October 2019}

\begin{document}

\maketitle

\section{Operadores}
    \begin{itemize}
        \item Operador Delta (\(\Delta\)):
            \begin{equation}
                \Delta\textit{f(t)} \equiv \textit{f(t+h) - f(t)}
            \end{equation}
    
        \item Operador E:
            \begin{equation}
                E\textit{f(t)} \equiv \textit{f(t+h)}
            \end{equation}
        \item Operador I:
            \begin{equation}
                I\textit{f(t)} \equiv \textit{f(t)}
            \end{equation}
    \end{itemize}

\section{Conclusões}
    Podemos perceber que ao combinar a equação (1) com a (2) e (3), nós vamos obter:
    \begin{equation}
        \Delta\textit{f(t)} = E\textit{f(t)} - I\textit{f(t)}
    \end{equation}
    \begin{equation}
        \Delta = E - I
    \end{equation}
    Outra conclusão seria:
    \begin{equation}
        E^k\textit{y} = y(x+kh)
    \end{equation}
    Podemos obter isto por meio de uma indução em k, tomando a base = 1.
    \[k = 1 \Longrightarrow E\textit{f(t)} = \Delta\textit{f(t)} + I\textit{f(t)} = \textit{f(t+h) - f(t) + f(t)} = \textit{f(t+h)} \]
    \[k = n \Longrightarrow E^n\textit{f(t)} = \textit{f(t + nh)}  \]
    \[E^{n+1}\textit{f(t)} = E(E^n\textit{f(t)}) = E(\textit{f(t + nh)}) = f(t + nh + h) = f(t + (n+1)h)\]
    
\section{Demonstração}
    Tomando \textit{y(x)} como:
    \begin{equation}
        \textit{y(x)} = \sum_{k = 0}^{n} a_k x^k
    \end{equation}
    Queremos provar que para todo \(n \in Z\), nós temos:
    \begin{equation}
        \Delta^n\textit{y(x)} = n!h^n a_n
    \end{equation}
    Uma rápida consequência disto seria que para todo j \(>\) n teríamos:
    \begin{equation}
        \Delta^j\textit{y(x)} = 0
    \end{equation}
    Isto é fácil de visualizar, pois:
    \[\Delta^{n+1}\textit{y(x)} = \Delta(\Delta^n\textit{y(x)} = \Delta(n!h^na_n) = n!h^na_n - n!h^na_n = 0\]
    Analogamente,
    \[\Delta(Constante) = 0\]
    Agora, vamos provar por meio de uma indução que a equação (8) é verdade. Vamos tomar como base k = 1.
    \[ \textit{y(x)} = a_0 + a_1x\]
    \[k = 1 \Longrightarrow \Delta^1\textit{y(x)} = \Delta\textit{y(x)} = \Delta(a_0+a_1x) = a_0 + a_1(x+h) - a_0 - a_1x = a_1h\]
    Tomando como verdade para n, vamos analisar o caso n+1:
    \[k = n+1 \Longrightarrow \textit{y(x)} = \sum_{k = 0}^{n+1} a_k x^k\]
    \[\Delta\textit{y(x)} = \Delta\sum_{k = 0}^{n+1} a_k x^k = \sum_{k = 0}^{n+1} \Delta(a_kx^k) = \sum_{k = 0}^{n+1} a_k(x+h)^k - \sum_{k = 0}^{n+1} a_kx^k\]
    \[= \sum_{k = 1}^{n+1} a_k\sum_{j = 1}^{k} {k \choose j}x^{k-j}h^j = \sum_{k = 1}^{n+1} a_k\sum_{j = 1}^{k} {k \choose j}x^{k-j}h^j\]
    \[= \sum_{k = 0}^{n} a_{k+1}\sum_{j = 0}^{k} {k+1 \choose j+1}x^{k-j}h^{j+1}\]
    Vamos visualizar isto de outra maneira:
    \[\sum_{k = 0}^{n} a_{k+1}\{{k+1 \choose 1}x^kh + {k+1 \choose 2} x^{n-1}h^{2}+...+ {k+1 \choose k}xh^k+h^{k+1}\}\]
    Ao aplicarmos a equação (9) da nossa hipótese de indução podemos perceber só existe um ter que não irá à 0:
    \[{n+1 \choose 1}a_{n+1}xh^n\]
    Vamos chamar tal termo de \textit{g(x)}, temos então:
    \[\Delta(\textit{g(x)} = g(x+h) - g(x) = {n+1 \choose 1} a_{n+1} \{(x+h)h^n - x^n\} = (n+1)!a_{n+1}h^{n+1}\]
    Portanto, comprovamos que a equação (8) é válida para y(x) na forma da equação (7).
\end{document}
