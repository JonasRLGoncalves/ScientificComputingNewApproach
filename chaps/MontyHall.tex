\documentclass{article}
\usepackage[utf8]{inputenc}
\usepackage[linguistics]{forest}

\title{Monty Hall}
\author{Jonas Gonçalves}
\date{October 2019}

\begin{document}

\maketitle

\section{Introduction}
    \quad The Monty Hall problem is a famous, seemingly paradoxical problem in conditional probability, loosely based on the game show \textit{Let's Make a Deal} and named after its original host, \textit{Monty Hall}.\par
    Suppose you're on a game show, being asked to choose between three doors. Behind one door is a car; behind the others, goats. You choose a door. The host opens one of the other doors, which he knows has a goat behind it. Then he asks whether you would like to switch your choice of door to the other remaining door.\par
    Is it your advantage to switch your choice?    

\section{Possible Outcomes}
    \quad One way to visualize the solution is to list all the possible outcomes. Without loss of generality, suppose your selection was door 2. Then the possible outcomes are:
    \begin{center}
        \begin{tabular}{ |c|c|c|c|c| }
            \hline
            Door 1 & \textbf{Door 2} & Door 3 & Stay & Switch \\
            \hline
            Car & Goat & Goat & Goat & \textbf{Car}\\
            \hline
            Goat & Car & Goat & \textbf{Car} & Goat\\
            \hline
            Goat & Goat & Car & Goat & \textbf{Car}\\
            \hline
        \end{tabular}
    \end{center}
    \quad We can see that by switching you have twice as much chance of winning than by staying with your original choice. \par
    
    Another way to visualize is by a decision tree:
    
    \begin{center}
        \begin{forest}
            [The host asks you to choose a door[The chosen door has a goat[You stay [You get a goat]][You change[You get a car]]][The chosen door has a car[You stay[You get a car]][You change[You get a goat]]][The chosen door has a goat [You stay[You get a goat]][You change[You get a car]]]]
        \end{forest}
    \end{center}
        
\section{Baye's Theorem}
    \quad Why aren’t the odds 50/50 after the host opens the door? Why does switching doors have a better chance of winning than staying with the first choice? Baye's Theorem can help us to  describe the probabilities. \par
    If A is the hypothesis ``door 2 has a car behind it'' and B is the evidence that Monty has revealed a door with a goat behind it. Then the problem can be rewritten as \(P(A|B)\), the conditional probability of A given B.
    
    \begin{equation}
        P(A|B) = \frac{P(B|A)P(A)}{P(B)} = \frac{P(B|A)P(A)}{P(B|A)P(A)+P(B|\overline{A})P(\overline{A})}, \overline{A} = notA
    \end{equation}
    
    \begin{itemize}
        \item P(A) is the prior probability that door 2 has a car behind it, without knowing about the door that Monty reveals. P(A) = \(\frac{1}{3}\).
        \item P(Ã) is the probability that not picking a door with a car behind it. P(Ã) = 1 - P(A) = \(\frac{2}{3}\).
        \item \(P(B|A)\) is the probability that Monty will show a goat, given that door 2 has a car. Since Monty always shows a door with a goat behind it, \(P(B|A)\) = 1.
        \item \(P(B|\overline{A})\) is the probability that Monty will show a goat, given that door 2 has a goat. Since Monty always shows a door with a goat behind it, \(P(B|\overline{A})\) = 1.
    \end{itemize}
    \[P(A|B) = \frac{1 \times \frac{1}{3}}{1 \times \frac{1}{3} + 1 \times \frac{2}{3}} = \frac{1}{3}\]
    The probability that the car is behind door 2 is completely unchanged. 
    
    \[P(\overline{A}|B) = \frac{P(B|\overline{A})P(\overline{A}}{P(B|A)P(A)+P(B|\overline{A})P(\overline{A})} = \frac{1 \times \frac{2}{3}}{1 \times \frac{1}{3} + 1 \times \frac{2}{3}} = \frac{2}{3}\]
    However, the probability it is behind the door that is not revealed is \(\frac{2}{3}\). Therefore, switching your original choice is twice as likely to get you the car.
\section{References}
    \begin{itemize}
        \item[\textbf{[1]}] https://brilliant.org/wiki/monty-hall-problem/
        \item[\textbf{[2]}] https://en.wikipedia.org/wiki/Monty\_Hall\_problem
    \end{itemize}
\end{document}
