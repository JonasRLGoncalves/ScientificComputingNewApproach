\documentclass{article}
\usepackage[utf8]{inputenc}
\usepackage{mathtools}

\title{Perceptron}
\author{Jonas Gonçalves}
\date{January 2020}

\begin{document}

\maketitle

\section{Introduction}
    
    The perceptron is a type of \textbf{Artificial Neural Network} (\textbf{ANN}), invented in 1958 at the Cornell Aeronautical Laboratory by Frank Reasonable. It can be seen as the simplest model of feedforward neural network: a linear classifier.
    
    In machine learning, the perceptron is an algorithm for supervised learning of binary classifiers, an algorithm that classifies elements of a given set into two groups on the basis of a classification rule. 
    
    One of its defining characteristics is that it's not just an iterative set of processes, but an evolving process where the machine learn from the data received over time.
    
\section{Definition}

    The perceptron is an algorithm for learning a binary classifier called a \textbf{threshold function}: a function that maps its input \textbf{x}, a real-valued vector, to an output value, a single binary value, \textit{f}(\textbf{x}):
    
    \[
        \textit{f}(\textbf{x}) =
        \begin{dcases}
            1, & \text{if } \mathbf{w} \cdot \mathbf{x} + b > 0\\
            0, & \text{otherwise}
        \end{dcases}
    \]
    
    where \textbf{w} is a vector of real-valued weights, \textbf{w} \(\cdot\) \textbf{x} is the product \(\sum_{i=1}^{m} w_i x_i\), where \textit{m} is the number of inputs to the perceptron, and \textbf{b} is the bias.
    
    The bias b doesn't depends on any input value, it shifts the decision boundary from the origin. 
    
    The image of the function, \textit{f}(\textbf{x}) is the set {0,1}, and those two different elements are utilized to classify \textbf{x} as either a positive or negative instance, in the case of a binary classification problem.
    
    The perceptron algorithm will never terminate if the learning set is not linearly separable, because the algorithm will never reach a point where all vectors are classified properly.
\section{Bibliography}
    \begin{itemize}
        \item[1.] https://en.wikipedia.org/wiki/Perceptron\#Learning\_algorithm
        \item[2.] https://pt.wikipedia.org/wiki/Perceptron
        \item[3.] https://en.wikipedia.org/wiki/Binary\_classification
        \item[4.] https://en.wikipedia.org/wiki/Supervised\_learning
        \item[5.] https://en.wikipedia.org/wiki/Feedforward\_neural\_network
        \item[6.] https://www.techopedia.com/definition/21333/perceptron
    \end{itemize}
\end{document}
